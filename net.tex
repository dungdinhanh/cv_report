\section{Networks}
\subsection{VGG}
This architecture is from VGG group, Oxford. It makes the improvement over AlexNet by replacing large kernel-sized filters (11 and 5 in the first and second convolutional layer, respectively) with multiple 3X3 kernel-sized filters one after another. With a given receptive field (the effective area size of input image on which output depends), multiple stacked smaller size kernel is better than the one with a larger size kernel because multiple non-linear layers increases the depth of the network which enables it to learn more complex features, and that too at a lower cost.


For example, three 3X3 filters on top of each other with stride 1 has a receptive size of 7, but the number of parameters involved is $3\times(9\times C^2)$ in comparison to $49\times C^2$ parameters of kernels with a size of 7. Here, it is assumed that the number of input and output channel of layers is C. Also, 3X3 kernels help in retaining finer level properties of the image. The network architecture is given in the table.

You can see that in VGG-D, there are blocks with same filter size applied multiple times to extract more complex and representative features. This concept of blocks/modules became a common theme in the networks after VGG.


The VGG convolutional layers are followed by 3 fully connected layers. The width of the network starts at a small value of 64 and increases by a factor of 2 after every sub-sampling/pooling layer. It achieves the top-5 accuracy of 92.3\% on ImageNet. As in [], VGG-D or VGG-16 performs better than others. Hence, in this project, we use VGG-16 as one of the method for classifying problem.

\subsection{Residual Network}
As per what we have seen so far, increasing the depth should increase the accuracy of the network, as long as over-fitting is taken care of. But the problem with increased depth is that the signal required to change the weights, which arises from the end of the network by comparing ground-truth and prediction becomes very small at the earlier layers, because of increased depth. It essentially means that earlier layers are almost negligible learned. This is called vanishing gradient. The second problem with training the deeper networks is, performing the optimization on huge parameter space and therefore naively adding the layers leading to higher training error. Residual networks allow training of such deep networks by constructing the network through modules called residual models as shown in the figure. This is called degradation problem. The intuition around why it works can be seen as follows:

Imagine a network, A which produces x amount of training error. Construct a network B by adding few layers on top of A and put parameter values in those layers in such a way that they do nothing to the outputs from A. Let’s call the additional layer as C. This would mean the same x amount of training error for the new network. So while training network B, the training error should not be above the training error of A. And since it DOES happen, the only reason is that learning the identity mapping (doing nothing to inputs and just copying as it is) with the added layers - C is not a trivial problem, which the solver does not achieve. To solve this, the module shown above creates a direct path between the input and output to the module implying an identity mapping and the added layer - C just need to learn the features on top of already available input. Since C is learning only the residual, the whole module is called residual module.
 
Also, similar to GoogLeNet, it uses a global average pooling followed by the classification layer. Through the changes mentioned, ResNets were learned with network depth of as large as 152. It achieves better accuracy than VGGNet and GoogLeNet while being computationally more efficient than VGGNet. ResNet-152 achieves 95.51\% top-5 accuracies.
 
The architecture is similar to the VGGNet consisting mostly of 3X3 filters. From the VGGNet, shortcut connection as described above is inserted to form a residual network. This can be seen in the figure which shows a small snippet of earlier layer synthesis from VGG-19.

